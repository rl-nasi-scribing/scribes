\section{Approximate Policy Iteration with a Parametric Approach}

High-level view of \emph{approximate policy iteration} (API) can be viewed as a specialized 
case of \emph{generalized policy iteration} (GPI) in Reinforcement Learning (RL). 
There are two main components:

\begin{enumerate}
    \item \textbf{Approximate Policy Evaluation (Approx ValIter):}
    This stage aims to evaluate the value function $V^\pi$ or $Q^\pi$ (depending 
    on whether we adopt a value-based or action-value-based approach) in an 
    \emph{approximate} manner.
    \begin{itemize}
        \item If a parametric approach is used, we define a value function 
        $\hat{V}(s; w)$ (or $\hat{Q}(s,a; w)$) with parameter $w \in \mathbb{R}^k$.
        \item The goal is to minimize the \emph{error} between $\hat{V}(s; w)$ 
        and the true value, often through a mean-squared error or a temporal difference error.
    \end{itemize}

    \item \textbf{Approximate Policy Improvement (Approx PolIter):}
    Once we have a sufficiently accurate approximation of the value function, we 
    perform \emph{policy improvement}—that is, we update the policy $\pi$ 
    (potentially also parameterized).
    \begin{itemize}
        \item For example, in a \emph{policy gradient} method, the policy 
        $\pi(a \mid s; w)$ itself can be a parametric function of $w$.
        \item In some scenarios, we might select an action $a$ that maximizes 
        $\hat{Q}(s,a; w)$ to yield a deterministic policy.
    \end{itemize}
\end{enumerate}

\noindent
\textbf{Parametric Approach.}
\begin{itemize}
    \item We assume there is a parameter vector $w \in \mathbb{R}^k$ that defines 
    the form of the value function or policy. For instance, if we use a linear 
    function approximation:
    \[
    \hat{V}(s; w) = \phi(s)^\top w,
    \]
    where $\phi(s) \in \mathbb{R}^k$ is a feature vector for state $s$.
    \item In some cases (e.g., a tabular setting), the dimensionality $k$ equals 
    the number of states, and $\phi(s)$ becomes a one-hot vector.
\end{itemize}

\noindent
\textbf{Parameter Optimization via SGD.}
\begin{itemize}
    \item We define an error function $e(w)$—for example, the mean-squared error 
    between $\hat{V}(s; w)$ and the target.
    \item We seek $w^*$ that minimizes $e(w)$:
    \[
    w^* 
    = \arg\min_{w} \; e(w).
    \]
    \item To achieve this, we often use \emph{stochastic gradient descent} (SGD) 
    or a related variant (e.g., Adam, RMSProp), updating $w$ iteratively:
    \[
    w \leftarrow w - \alpha \,\nabla_w e(w),
    \]
    where $\alpha$ is the learning rate.
\end{itemize}

\noindent
\textbf{Iterative Process in RL.}
\begin{itemize}
    \item In RL, approximate evaluation and approximate improvement typically 
    occur iteratively while the agent gathers new data by interacting with the environment.
    \item Each time the value function is updated, the policy can be improved, 
    and the new policy is then evaluated again, and so forth. This is the essence 
    of \emph{generalized policy iteration} (GPI).
\end{itemize}
